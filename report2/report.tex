\documentclass[a4paper,12pt,oneside,final]{report}
%\usepackage{geometry}                % See geometry.pdf to learn the layout options. There are lots.
%\geometry{landscape}                % Activate for for rotated page geometry
%\usepackage[parfill]{parskip}    % Activate to begin paragraphs with an empty line rather than an indent
\usepackage{graphicx}
% \usepackage{amssymb}
\usepackage{epstopdf}
\usepackage[utf8]{inputenc}
\usepackage{titlesec}
\usepackage[titletoc]{appendix}
\titleformat{\chapter}[hang]{\bf\Huge}{\thechapter}{1cm}{}

\pagestyle{plain}
% -------------------- this stuff for code --------------------

\usepackage{anysize}
\marginsize{30mm}{30mm}{20mm}{20mm}

\newenvironment{formal}{%
  \def\FrameCommand{%
    \hspace{1pt}%
    {\color{blue}\vrule width 2pt}%
    {\color{formalshade}\vrule width 4pt}%
    \colorbox{formalshade}%
  }%
  \MakeFramed{\advance\hsize-\width\FrameRestore}%
  \noindent\hspace{-4.55pt}% disable indenting first paragraph
  \begin{adjustwidth}{}{7pt}%
  \vspace{2pt}\vspace{2pt}%
}
{%
  \vspace{2pt}\end{adjustwidth}\endMakeFramed%
}

\newenvironment{changemargin}[2]{\begin{list}{}{%
\setlength{\topsep}{0pt}%
\setlength{\leftmargin}{0pt}%
\setlength{\rightmargin}{0pt}%
\setlength{\listparindent}{\parindent}%
\setlength{\itemindent}{\parindent}%
\setlength{\parsep}{0pt plus 1pt}%
\addtolength{\leftmargin}{#1}%
\addtolength{\rightmargin}{#2}%
}\item }{\end{list}}

\usepackage{color}
\usepackage{dsfont}
\usepackage[bitstream-charter]{mathdesign}
\usepackage[scaled]{helvet}
\usepackage{inconsolata}


\definecolor{colKeys}{rgb}{0,0,0.9} 
\definecolor{colIdentifier}{rgb}{0,0,0} 
\definecolor{colString}{rgb}{0.7,0,0} 
\definecolor{colComments}{rgb}{0,0.6,0} 
\usepackage{listings}
\lstset{
  language=python,
  stringstyle=\color{colString},
  keywordstyle=\color{colKeys},
  identifierstyle=\color{colIdentifier},
  commentstyle=\color{colComments},
  numbers=left,
  tabsize=4,
  frame=single,
  breaklines=true,
  basicstyle=\small\ttfamily,
  numberstyle=\tiny\ttfamily,
  framexleftmargin=0mm,
  xleftmargin=7mm,
  xrightmargin=7mm,
  frameround={tttt},
  captionpos=b
}

%% Headers and footers
\usepackage{fancyhdr}
\usepackage[section]{placeins}
\pagestyle{fancy}
\fancyhf{}
\addtolength{\headwidth}{30pt}
\addtolength{\headwidth}{30pt}
\renewcommand{\headrulewidth}{0.4pt} % thickness of the header line
\renewcommand{\footrulewidth}{0.4pt} % thickness of the footer line
\renewcommand{\chaptermark}[1]{\markboth{#1}{#1}} % chapter name
\renewcommand{\sectionmark}[1]{\markright{\thesection\ #1}}  % section name
\lhead[\fancyplain{}{\bf\thepage}]{\fancyplain{}{\bf\rightmark}} % display header
\rhead[\fancyplain{}{\bf\leftmark}]{\fancyplain{}{}} % display header
\fancyfoot[C]{\bf\thepage} % display footer (page number)
\fancyfoot[R]{\bf\today} % display footer (date)
\fancypagestyle{plain}{ 
	\fancyhead{} \renewcommand{\headrulewidth}{0pt}
}
\newcommand{\clearemptydoublepage}{\newpage{\pagestyle{plain}\cleardoublepage}}

\usepackage[T1]{fontenc}
\usepackage{enumerate}
\usepackage{afterpage,lastpage,fancyhdr}
\usepackage[includeheadfoot,margin=2.5cm]{geometry}
\geometry{letterpaper}                   % ... or a4paper or a5paper or ... 

% -------------------- end of code stuff --------------------



\DeclareGraphicsRule{.tif}{png}{.png}{`convert #1 `dirname #1`/`basename #1 .tif`.png}

\makeatletter \def\thickhrulefill{\leavevmode \leaders \hrule height 1pt\hfill
\kern \z@} \renewcommand{\maketitle}{
    \begin{titlepage}
    \let\footnotesize\small \let\footnoterule\relax \parindent \z@ \reset@font
    \null\vfil
    \vspace{-20mm}
    \begin{center}
    {\small \scshape Imperial College London}
    \end{center}
    \vspace{0.5cm}
	\begin{minipage}{\textwidth}
		\vspace{1cm}
		%\noindent\rule[0ex]{\textwidth}{4pt} \\
		%\flushright
		\center
		\@title
		\\ \vspace{4mm}
		%\noindent\rule[0ex]{\textwidth}{4pt} \\
	\end{minipage}
	\vspace{2cm}
	\begin{center}
		\includegraphics[width=70mm,]{logo_imperial_college_london.png}
	\end{center}
	\vspace{5.4cm}
	\vspace{\stretch{1}}
	\begin{minipage}{\textwidth}
		\flushright
		{\bfseries}
		\vspace{7mm}
		\center
		\@author\\
	\end{minipage}
	\vspace{20mm}
		\flushleft
		{\bfseries}
		{\small \scshape \@date }
		\vspace{0.1cm}
		\rule{\linewidth}{.5pt}
  \end{titlepage}
  \setcounter{footnote}{1}
  \setcounter{page}{2}
}


\author{Paul Gribelyuk (pg1312, a5)}
\makeatother
\title{\Huge \#DOC417 - Advanced Computer Graphics Coursework 2}
\date{\today}

\begin{document}
\maketitle
% \tableofcontents
\listoffigures
\chapter{Generating Plots of Fresnel Reflectance}
\paragraph{}
I produced the graphs of Fresnel reflectance using iPython with the following scripts:
\begin{lstlisting}[ ]
def snell_thetaT(thetaI, etaI, etaT):
    return asin(sin(thetaI)*etaI/etaT)

def Rs(thetaI, etaI, etaT):
    thetaT = snell_thetaT(thetaI, etaI, etaT)
    numer = etaI * cos(thetaI) - etaT * cos(thetaT)
    denom = etaI * cos(thetaI) + etaT * cos(thetaT)
    return abs(numer / denom)**2

def Rp(thetaI, etaI, etaT):
    thetaT = snell_thetaT(thetaI, etaI, etaT)
    numer = etaI * cos(thetaT) - etaT * cos(thetaI)
    denom = etaI * cos(thetaT) + etaT * cos(thetaI)
    return abs(numer / denom)**2
\end{lstlisting}
The graph associated the cofficients of parallel and perpendicular reflection, $R_p$ and $R_s$ respectively, for light entering a dielectric material (with $\eta = 1.5$) is displayed in Figure \ref{fig:fresnel1}.
\begin{figure}[!h]
  \begin{changemargin}{-50mm}{-50mm}
    \center
    \includegraphics[scale=0.8]{fresnel1.png}
    \caption{Light entering a dielectric medium \label{fig:fresnel1}}
  \end{changemargin}
\end{figure}

The graph associated with $R_p$ and $R_s$ for light leaving a dielectric with $\eta = 1.5$ is displayed in Figure \ref{fig:fresnel2}.
\begin{figure}[!h]
  \begin{changemargin}{-50mm}{-50mm}
    \center
    \includegraphics[scale=0.8]{fresnel2.png}
    \caption{Light leaving a dielectric medium \label{fig:fresnel2}}
  \end{changemargin}
\end{figure}
The Brewster angle, was computed to be the point where $R_p = 0$, was calculated from the plotted data to be $0.981158$ radians (56.22 degrees).  The exact value calculated from the formula $\theta_B = \arctan\left(\frac{\eta_t}{\eta_i}\right)$ is $0.981443$ radians (56.23 degrees).  The critical angle was calculated directly from the formula, $\theta_C = \arcsin\left(\frac{\eta_t}{\eta_i}\right)$, to be $0.729728$ (41.81 degrees).

\chapter{Sampling an Environment Map}
In this section I sample the Grace Cathedral environment map (EM) using the cumulative density function (CDF) as well as the analytical Phong model.
\section{Monte Carlo Sampling using CDF Inversion}
This method assumes that the luminance of the pixels forms a 2D probability density from which we can sample.  The approach is as follows:
\begin{itemize}
  \item[1] Compute average luminance in each row of the image forming a 1D probability density, \verb+rowAverages+
  \item[2] Perform CDF inversion on 1D array
  \begin{itemize}
    \item[a] Calculate $CDF(i) = ( \sum_{j=0}^{i-1} array[j] ) / (\sum_{j=0}^{len(array)} array[j])$ and $CDF(0) = 0$
    \item[b] Draw random variate $\xi$
    \item[c] Find index $k$ such that $CDF[k] \geq \xi$ and $CDF[k-1] < \xi$ via binomial search, since $CDF$ is sorted
    \item[d] $k$ is the sample according to the 1D PDF represented by the input array
  \end{itemize}
  \item[3] Perform the same 1D sampling on the selected row represented by $k$, to get another index $l$
  \item[4] The pair $(k, l)$ represent the index values of a CDF-based sample of the 2D image
\end{itemize}

I applied this technique on the Grace Cathedral environment map, with images, displayed in Figures \ref{fig:grace_cdf_64}, \ref{fig:grace_cdf_256}, \ref{fig:grace_cdf_1024}

\begin{figure}[!h]
  \begin{changemargin}{-50mm}{-50mm}
    \center
    \includegraphics[scale=0.4]{grace_cdf_64.png}
    \caption{Grace Cathedral sampled via CDF Inversion with 64 samples \label{fig:grace_cdf_64}}
  \end{changemargin}
\end{figure}

\begin{figure}[!h]
  \begin{changemargin}{-50mm}{-50mm}
    \center
    \includegraphics[scale=0.4]{grace_cdf_64.png}
    \caption{Grace Cathedral sampled via CDF Inversion with 256 samples \label{fig:grace_cdf_256}}
  \end{changemargin}
\end{figure}

\begin{figure}[!h]
  \begin{changemargin}{-50mm}{-50mm}
    \center
    \includegraphics[scale=0.4]{grace_cdf_1024.png}
    \caption{Grace Cathedral sampled via CDF Inversion with 1024 samples \label{fig:grace_cdf_1024}}
  \end{changemargin}
\end{figure}

\section{Phong Lobe Sampling}
The Phong sampling technique is an analytical procedure, not relying on the input image data.  The formulation is as follows (exponent $n$ is an input):
\begin{itemize}
  \item[1] Draw uniform random variates $\xi_1$ and $\xi_2$
  \item[2] The corresponding sampled angular values are $(\theta, \phi) = \left(\cos^{-1}\left\{(1 - \xi_1)^{\frac{1}{n+1}}\right\}, 2\pi\xi_2\right)$
  \item[3] The corresponding indices in the latlong map are $(k, l) = \left(\frac{2\theta}{\pi} \cdot H, \frac{\phi}{2\pi} \cdot W\right)$
\end{itemize}
The images resulting from the application of this tecnique are in Figures \ref{fig:phong_256_10} and \ref{fig:phong_1024_50}.
\begin{figure}[!h]
  \begin{changemargin}{-50mm}{-50mm}
    \center
    \includegraphics[scale=0.4]{grace_phong_256_10.png}
    \caption{Grace Cathedral sampled via Phong with 256 samples and $n=5$ \label{fig:phong_256_10}}
  \end{changemargin}
\end{figure}

\begin{figure}[!h]
  \begin{changemargin}{-50mm}{-50mm}
    \center
    \includegraphics[scale=0.4]{grace_phong_1024_50.png}
    \caption{Grace Cathedral sampled via Phong with 1024 samples and $n=50$ \label{fig:phong_1024_50}}
  \end{changemargin}
\end{figure}



\end{document}  
